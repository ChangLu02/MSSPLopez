---
title: "T_test"
author: "Qiuyi Feng"
date: "2025-02-20"
output:
  pdf_document: default
  html_document: default
---

\begin{verbatim}
knitr::opts_chunk$set(echo = TRUE)
\end{verbatim}

\begin{verbatim}
df_students<-read.csv('df_students.csv')
\end{verbatim}
\begin{verbatim}
df_students
\end{verbatim}
Difference Between U-Test and T-Test
1. Definition & Purpose
T-Test: A parametric test used to compare the means of two groups to determine if they are significantly different. It assumes that the data follows a normal distribution.
U-Test (Mann-Whitney U Test): A non-parametric test used to compare differences between two independent groups when the data does not necessarily follow a normal distribution.
2. Underlying Principle
T-Test: Based on the assumption that the sample data is normally distributed and uses means and variances to assess differences between groups.
U-Test: Ranks all observations from both groups together and compares the sum of ranks instead of actual data values.




Since U-test and T-test only focus on binary data, so some category data like 'Age Arrive USA', continous data 'quiz time(categorized by 10 minutes)' don't work.


1. quiz time(categorized by 10 minutes) //We didn't test because it is not binary variable.//

2. English Proficiency

3. Race

4. Home Language

#### Mann-Whitney U Test

1. Accomodations

2. Age Arrive USA //We didn't test because it is not binary variable.//


3. Sex

4. Born in USA


In T and U test, We only consider whether P-value is smaller than 0.05 or not. If P-value is greater than 0.05, these two different groups have no such a significant difference in the persepective of response variable.

T-test
1.English Proiciency
\begin{verbatim}
group1 <- df_students$\texttt{Total.Score}[df_students$\texttt{English.Proficiency} == 0]
group2 <- df_students$\texttt{Total.Score}[df_students$\texttt{English.Proficiency} == 1]

t_test_result <- t.test(group1, group2, var.equal = TRUE)  
print(t_test_result)
\end{verbatim}

2.Race
\begin{verbatim}
group1 <- df_students$\texttt{Total.Score}[df_students$\texttt{Race.Ethnicity} == 0]
group2 <- df_students$\texttt{Total.Score}[df_students$\texttt{Race.Ethnicity} == 1]

t_Race <- t.test(group1, group2, var.equal = TRUE)  
print(t_Race)
\end{verbatim}
3.Home Language
\begin{verbatim}
group1 <- df_students$\texttt{Total.Score}[df_students$\texttt{Home.Language} == 0]
group2 <- df_students$\texttt{Total.Score}[df_students$\texttt{Home.Language} == 1]

t_HomeLanguage <- t.test(group1, group2, var.equal = TRUE)  
print(t_HomeLanguage)
\end{verbatim}
4.Sex
\begin{verbatim}
group1 <- df_students$\texttt{Total.Score}[df_students$Sex == 0]
group2 <- df_students$\texttt{Total.Score}[df_students$Sex == 1]

t_test_result <- t.test(group1, group2, var.equal = TRUE)  
print(t_test_result)
\end{verbatim}

5.Born USA
\begin{verbatim}
group1 <- df_students$\texttt{Total.Score}[df_students$\texttt{Born.USA} == 0]
group2 <- df_students$\texttt{Total.Score}[df_students$\texttt{Born.USA} == 1]

t_test_result <- t.test(group1, group2, var.equal = TRUE)  
print(t_test_result)
\end{verbatim}


6.Accomodations
\begin{verbatim}
group1 <- df_students$\texttt{Total.Score}[df_students$Accomodations == 0]
group2 <- df_students$\texttt{Total.Score}[df_students$Accomodations == 1]

t_test_result <- t.test(group1, group2, var.equal = TRUE)  
print(t_test_result)
\end{verbatim}

U-test
1.Sex
\begin{verbatim}
u_test_sex <- wilcox.test(df_students$\texttt{Total.Score} ~ df_students$Sex, exact = FALSE)

print(u_test_sex)

\end{verbatim}
Sex is 0.139, which is not verified to the U test from client.

2.Born USA
\begin{verbatim}

u_test_born <- wilcox.test(df_students$\texttt{Total.Score} ~ df_students$\texttt{Born.USA}, exact = FALSE)

print(u_test_born)

\end{verbatim}
The p-value of Born in USA is also different 
3.Accomodations
\begin{verbatim}
u_test_accomodations <- wilcox.test(df_students$\texttt{Total.Score} ~ df_students$Accomodations, exact = FALSE)

print(u_test_accomodations)

\end{verbatim}

4.English 
\begin{verbatim}

u_test_english <- wilcox.test(df_students$\texttt{Total.Score} ~ df_students$\texttt{English.Proficiency}, exact = FALSE)

print(u_test_english)

\end{verbatim}

5.Race
\begin{verbatim}

u_test_race <- wilcox.test(df_students$\texttt{Total.Score} ~ df_students$\texttt{Race.Ethnicity}, exact = FALSE)

print(u_test_race)

\end{verbatim}

6.Home language
\begin{verbatim}

u_test_homelanguage <- wilcox.test(df_students$\texttt{Total.Score} ~ df_students$\texttt{Home.Language}, exact = FALSE)

print(u_test_homelanguage)

\end{verbatim}
In order to get a comprehensive result, We also calculated  t test for those none-normal distributed and the u test for those normal distributed data.


1.race_T
\begin{verbatim}
group1 <- df_students$\texttt{Quiz.Time}[df_students$\texttt{English.Proficiency} == 0]
group2 <- df_students$\texttt{Quiz.Time}[df_students$\texttt{English.Proficiency} == 1]
t_test_result <- t.test(group1, group2, var.equal = TRUE)  
print(t_test_result)
\end{verbatim}

2.Enlgish_T
\begin{verbatim}
group1 <- df_students$\texttt{Quiz.Time}[df_students$\texttt{English.Proficiency} == 0]
group2 <- df_students$\texttt{Quiz.Time}[df_students$\texttt{English.Proficiency} == 1]

t_test_result <- t.test(group1, group2, var.equal = TRUE)  
print(t_test_result)
\end{verbatim}

3.BornUS_T
\begin{verbatim}
group1 <- df_students$\texttt{Quiz.Time}[df_students$\texttt{English.Proficiency} == 0]
group2 <- df_students$\texttt{Quiz.Time}[df_students$\texttt{English.Proficiency} == 1]

t_test_result <- t.test(group1, group2, var.equal = TRUE)  
print(t_test_result)
\end{verbatim}

1.race_U
\begin{verbatim}

u_test_race_quiztime <- wilcox.test(df_students$\texttt{Quiz.Time} ~ df_students$\texttt{Race.Ethnicity}, exact = FALSE)

print(u_test_race_quiztime)

\end{verbatim}
2.English_U
\begin{verbatim}

u_test_english_quiztime <- wilcox.test(df_students$\texttt{Quiz.Time} ~ df_students$\texttt{English.Proficiency}, exact = FALSE)

print(u_test_english_quiztime)

\end{verbatim}
3.BornUS_U
\begin{verbatim}

u_test_born_quiztime <- wilcox.test(df_students$\texttt{Quiz.Time} ~ df_students$\texttt{Born.USA}, exact = FALSE)

print(u_test_born_quiztime)

\end{verbatim}

\begin{verbatim}
chisq.test(table(df_students$\texttt{Quiz.Time.Group}, df_students$\texttt{English.Proficiency}))
\end{verbatim}


We can find in all those different grouping method, the P-value is larger than 0.05, which means  there is no such a huge difference between diffient groups.
